%%% Copyright (C) 2020 David Beauchemin
%%%
%%% Ce fichier et tous les fichiers .tex dont la racine est
%%% mentionnée dans les commandes \include et \input ci-dessous font
%%% partie du projet «Reproductibilité en apprentissage automatique
%%% - Webinaire de l'institut intelligence et données»
%%% URL
%%%
%%% Le format et le visuel est très fortement inspiré du matériel de
%%% Vincent Goulet https://gitlab.com/vigou3/webinaire-recherche-reproductible
%%%
%%% Cette création est mise à disposition selon le contrat
%%% Attribution-Partage dans les mêmes conditions 4.0
%%% International de Creative Commons.
%%% https://creativecommons.org/licenses/by-sa/4.0/

\documentclass[aspectratio=169,10pt,xcolor=x11names,english,french]{beamer}
\usepackage{babel}
\usepackage[autolanguage]{numprint}
\usepackage{amsmath}
\usepackage{currfile}                  % nom fichier de script
\usepackage{changepage}                % page licence
\usepackage{tabularx}                  % page licence
\usepackage{relsize}                   % \smaller et al.
\usepackage{awesomebox}                % boites signalétiques
\usepackage{fancyvrb}                  % texte verbatim
\usepackage{framed}                    % env. leftbar
\usepackage{pict2e}                    % cycle travail Git
\usepackage[overlay,absolute]{textpos} % couvertures
\usepackage{metalogo}       
\usepackage{textpos}           % logo \XeLaTeX

\usepackage{fontawesome}

\usepackage{stackengine} % for stacking logo

%% =============================
%%  Informations de publication
%% =============================
\title{Reproductibilité en apprentissage automatique}
\author{David Beauchemin}
\date{30 octobre 2020}
\newcommand{\reposurl}{...}

%% =======================
%%  Apparence du document
%% =======================

%% Thème Beamer général
\usetheme[titleformat=allcaps, numbering=none, block=transparent]{metropolis}

%% Modifications aux couleurs: fond blanc, titres en texte noir sur
%% fond blanc
\setbeamercolor{normal text}{bg=white}
\setbeamercolor{frametitle}{fg=normal text.fg, bg=}

%% Déplacer les titres vers le bas sous la décoration du gabarit CFDD
\makeatletter
\setlength{\metropolis@frametitle@padding}{4.9ex}
\renewcommand{\metropolis@frametitlestrut@start}{
	\rule{0pt}{6ex +%
		\totalheightof{%
			\ifcsdef{metropolis@frametitleformat}{\metropolis@frametitleformat X}{X}%
		}%
	}
}
\renewcommand{\metropolis@frametitlestrut@end}{}
\makeatother

%% Format de la page de titre de section;
\makeatletter
\setbeamertemplate{section page}{%
	\begin{textblock*}{92.5mm}[1,1](160mm,90mm)
		\includegraphics[height=\paperheight,keepaspectratio]{img/Élément2_PPT}
	\end{textblock*}
	\begin{minipage}{22em}
		\raggedright
		\usebeamerfont{section title}
		\let\hyperlink\@secondoftwo\textcolor[cmyk]{0.67, 0.66, 0, 0.71}{\insertsectionhead}\par
	\end{minipage}
	\par
	\vspace{\baselineskip}
}
\makeatother

%% Polices de caractères
\newfontfamily\Overpass{Overpass}
\setsansfont{Overpass}        % police principale
\newfontfamily\OverpassSemiBold{Overpass}
[
BoldFont = *-SemiBold
]
\newfontfamily\OverpassExtraLightBold{Overpass}
[
UprightFont = *-ExtraLight,
BoldFont = *-Light
]
\newfontfamily\OverpassLight{Overpass}
[
UprightFont = *-Light,
BoldFont = *-Regular
]

%% Couleurs additionnelles
\definecolor{comments}{cmyk}{0,1, 1, 0}   % rouge foncé
\definecolor{link}{cmyk}{0.67, 0.66, 0, 0.71}    % liens internes
\definecolor{url}{cmyk}{0,1, 1, 0}       % liens externes
\definecolor{codebg}{named}{LightYellow1} % fond code R
\definecolor{rouge}{rgb}{0.85,0,0.07} % bandeau rouge UL
\definecolor{or}{rgb}{1,0.8,0}        % bandeau or UL
\colorlet{alert}{mLightBrown} % alias pour couleur Metropolis
\colorlet{dark}{mDarkTeal}    % alias pour couleur Metropolis
\colorlet{code}{mLightGreen}  % alias pour couleur Metropolis
\colorlet{shadecolor}{codebg}


\definecolor{titre}{cmyk}{0.67, 0.66, 0, 0.71}
\setbeamercolor{frametitle}{fg=titre}
\setbeamercolor{section in head/foot}{fg=titre}

%% Hyperliens
\hypersetup{%
	pdfauthor = {David Beauchemin},
	pdftitle = {Reproductibilité en apprentissage automatique},
	colorlinks = {true},
	linktocpage = {true},
	allcolors = {link},
	urlcolor = {url},
	pdfpagemode = {UseOutlines},
	pdfstartview = {Fit},
	bookmarksopen = {true},
	bookmarksnumbered = {true},
	bookmarksdepth = {subsection}}

%% Paramétrage de babel pour les guillemets
\frenchbsetup{og=«, fg=»}

%% =========================
%%  Nouveaux environnements
%% =========================

%% Environnement pour le code informatique; hybride
%% des environnements snugshade* et leftbar de framed.
\makeatletter
\newenvironment{Scode}{%
	\def\FrameCommand##1{\hskip\@totalleftmargin
		\vrule width 3pt\colorbox{codebg}{\hspace{5pt}##1}%
		% There is no \@totalrightmargin, so:
		\hskip-\linewidth \hskip-\@totalleftmargin \hskip\columnwidth}%
	\MakeFramed {\advance\hsize-\width
		\@totalleftmargin\z@ \linewidth\hsize
		\advance\labelsep\fboxsep
		\@setminipage}%
}{\par\unskip\@minipagefalse\endMakeFramed}
\makeatother

%% Environnement pour le contenu d'un fichier; alias de snugshade*.
\newenvironment{Sfile}{\begin{snugshade*}}{\end{snugshade*}}

%% =====================
%%  Nouvelles commandes
%% =====================

%% Lien externe
\newcommand{\link}[2]{\href{#1}{#2~{\smaller\faExternalLink*}}}

%% Simili commande \HUGE
\newcommand{\HUGE}{\fontsize{36}{36}\selectfont}

%% Raccourcis usuels vg
\newcommand{\code}[1]{\textcolor{code}{\texttt{#1}}}

%%% =======
%%%  Varia
%%% =======

%% Longueurs pour la composition des pages couvertures avant et
%% arrière
\newlength{\banderougewidth} \newlength{\banderougeheight}
\newlength{\bandeorwidth}    \newlength{\bandeorheight}
\newlength{\imageheight}     \newlength{\imagewidth}
\newlength{\logoheight}

\begin{document}
	
	%% Style de l'entête et du pied de page
	\setbeamertemplate{headline}{%
		\ifnum\insertframenumber>1\relax%
		\begin{textblock*}{17mm}(0mm,0mm)
			\includegraphics[width=\linewidth,keepaspectratio]{img/Élément4_PPT}
		\end{textblock*}
		\fi}
	\setbeamertemplate{footline}{%
		\ifnum\insertframenumber>1\relax%
		\begin{textblock*}{24.5mm}[1,1](160mm,90mm)
			\includegraphics[width=\linewidth,keepaspectratio]{img/Élément3_PPT}
		\end{textblock*}
		\fi}
	
	%% frontmatter
	\input{couverture-avant}
	
	\begin{frame}{Objectifs de la présentation}
		\begin{itemize}
			\item Sensibiliser sur les enjeux de la reproductibilité.
			\item Inciter l'intégration des solutions permettant une meilleure reproductibilité dans vos solutions d'affaires ou académiques.
			\item Améliorer votre productivité.
		\end{itemize}
	\end{frame}
	
	\begin{frame}
		\frametitle{Votre conférencier}
		
		\begin{minipage}{0.25\linewidth}
			\includegraphics[width=\linewidth,keepaspectratio]{img/david}
		\end{minipage}
		\hfill
		\begin{minipage}{0.70\linewidth}
			\begin{itemize}
				\item Introduit à la recherche reproductible en 2016 (R Markdown et Git)
				\item Participation à REPROLANG de la conférence LREC \cite{garneau2020robust}
				\item Membre actif dans le développement d'une librairie facilitant la reproductibilité (\href{https://poutyne.org/}{Poutyne})
			\end{itemize}
		\end{minipage}
		
		\begin{minipage}{0.25\linewidth}
			\small
			\textbf{DAVID BEAUCHEMIN} \\
			Candidat au doctorat \\
			Département d'informatique et de génie logiciel
		\end{minipage}
	\end{frame}
	
	\section{Introduction}
	\begin{frame}{C'est quoi la reproductibilité ?}
		La reproductibilité est le principe qu'on ne peut tirer de conclusions que d'un événement bien décrit, qui est apparu plusieurs fois, provoqué par des \textbf{personnes différentes}.
		
		Toutefois, ont utilise souvent ce terme pour spécifiquement désigner la \textbf{réplicabilité}. Soit la réplication (reproduction) des résultats d'un article dans des environnements pas (toujours) différents \cite{replicationvsreproductiblity, pineau2020improving}.
	\end{frame}
%La définition de reproductibilité est à revoir. Je pense que les points clés c'est l'idée que si une idée a de bons résultats, on devrait être capable de la reprendre et de retrouver les mêmes résultats. En ML, les idées c'est des algos et la reproductibilité devient juste d'être capable de s'assurer que les performances rapportées sont les mêmes. Aussi, la parenthèse avec la réplicabilité : la réplicabilité en ML le point clé c'est de réimplémenter un algo alors que la reproductibilité c'est juste genre rerouler le code from scratch et retrouver les perfos rapportées.
	
	\begin{frame}{Au menu}
			\uncover<1->{\begin{minipage}{0.24\linewidth}
				\centering
				\fontsize{35}{35}\selectfont\textcolor[cmyk]{0.67, 0.66, 0, 0.71}\faCopy \vfil
				\normalsize Répliquer
			\end{minipage}}
			\uncover<2->{\begin{minipage}{0.24\linewidth}
				\centering
				\fontsize{35}{35}\selectfont\textcolor[cmyk]{0.67, 0.66, 0, 0.71}\faDatabase\vfil
				\normalsize Jeux de données
				
			\end{minipage}}
				\uncover<3->{\begin{minipage}{0.24\linewidth}
					\centering
					\fontsize{35}{35}\selectfont\textcolor[cmyk]{0.67, 0.66, 0, 0.71}\faCogs\vfil
					\normalsize Procédure d'entraînement
			\end{minipage}}
				\uncover<4->{\begin{minipage}{0.24\linewidth}
					\centering
					\fontsize{35}{35}\selectfont\textcolor[cmyk]{0.67, 0.66, 0, 0.71}\faCode\vfil
					\normalsize Code
			\end{minipage}}
		\note{
			
		Être capable de \textbf{répliquer} les résultats d'un article d'un projet, 
	
		à partir du même \textbf{jeu de données} ou un jeu de données différent (mais proche), 
		
		en utilisant la \textbf{procédure d'entraînement} de l'article ou en utilisant notre procédure d'entrainement et 
		
		en utilisant le \textbf{code} du projet.}
	\end{frame}
	
	\begin{frame}{Pourquoi s'y intéresser?}
		\centering
		\fontsize{35}{35}\selectfont\textcolor[cmyk]{0.67, 0.66, 0, 0.71} 70~\%\footnote[1]{\cite{baker500ScientistsLift2016}}
		
		\note{
			$70$~\% des chercheurs en science on échoué dans leur tentative de reproduire un article d'un autre chercheur
		}
	\end{frame}

	\begin{frame}{Pourquoi s'y intéressé?}
	\centering
	\fontsize{35}{35}\selectfont\textcolor[cmyk]{0.67, 0.66, 0, 0.71} 50~\%\footnote[1]{\cite{baker500ScientistsLift2016}}

		
		\note{
			$50$~\% n'ont pas réussit à reproduire leurs \textbf{propres} expérimentations
			
			L'informatique ne fait pas exception à cela malgré la simplicité (théorique) de réplication des résultats. Selon une étude, sur \textbf{255} articles, près de $40$~\% n'étaient pas réplicable \cite{raff2019step}.
		}
	
	\end{frame}

	\begin{frame}{Pourquoi s'y intéressé?}
		\centering
		\fontsize{35}{35}\selectfont\textcolor[cmyk]{0.67, 0.66, 0, 0.71} 40~\%\footnote[2]{\cite{raff2019step}}
		
		
		\note{
			L'informatique ne fait pas exception à cela malgré la simplicité (théorique) de réplication des résultats. Selon une étude, sur \textbf{255} articles, près de $40$~\% n'étaient pas réplicable \cite{raff2019step}.
		}
		
	\end{frame}
	
	\begin{frame}{Motivation}
		\centering
		\uncover<1->{\begin{minipage}{0.24\linewidth}
				\centering
				\fontsize{35}{35}\selectfont\textcolor[cmyk]{0.67, 0.66, 0, 0.71}\faRecycle\vfil
				\normalsize Réutilisation
		\end{minipage}}
		\uncover<2->{\begin{minipage}{0.24\linewidth}
				\centering
				\fontsize{35}{35}\selectfont\textcolor[cmyk]{0.67, 0.66, 0, 0.71}\faForward\vfil
				\normalsize Productivité
		\end{minipage}}
		\uncover<3->{\begin{minipage}{0.24\linewidth}
				\centering
				\fontsize{35}{35}\selectfont\textcolor[cmyk]{0.67, 0.66, 0, 0.71}\faShare\vfil
				\normalsize Transfert
		\end{minipage}}
		\uncover<4->{\begin{minipage}{0.24\linewidth}
				\centering
				\fontsize{35}{35}\selectfont\textcolor[cmyk]{0.67, 0.66, 0, 0.71}\faMapPin\vfil
				\normalsize Se faire connaître
		\end{minipage}}
		\note{La reproductibilité facilite la \textbf{réutilisation} pour d'autres projets de recherche, améliorer votre productivité \textbf{et} permet le transfert vers l'industrie (plus facilement). En bonus, cela aide à vous faire connaître si vous faites du contenu réutilisable par la communauté.}
	\end{frame}
	
	\section{Les barrières à la réplicabilité}
		\begin{frame}		\def\ban{\fontsize{65}{65}\selectfont\textcolor[cmyk]{0.67, 0.66, 0, 0.71}\faBan}
		\def\database{\fontsize{35}{35}\selectfont\textcolor[cmyk]{0.67, 0.66, 0, 0.71}\faDatabase}
		\def\code{\fontsize{35}{35}\selectfont\textcolor[cmyk]{0.67, 0.66, 0, 0.71}\faCode}
		\def\gear{\fontsize{35}{35}\selectfont\textcolor[cmyk]{0.67, 0.66, 0, 0.71}\faCog}
		\centering
		\uncover<1->{\begin{minipage}{0.24\linewidth}
			\centering
			\def\stackalignment{c}
			\topinset{\database}{\rotatebox{90}{\ban}}{14pt}{-1pt}\vfil
			\normalsize Non-disponibilité des données
		\end{minipage}}
		\uncover<2->{\begin{minipage}{0.24\linewidth}
				\centering
				\fontsize{35}{35}\selectfont\textcolor[cmyk]{0.67, 0.66, 0, 0.71}\faCogs\vfil
				\normalsize Mauvaise spécification
		\end{minipage}}
		\uncover<3->{\begin{minipage}{0.24\linewidth}
				\centering
				\def\stackalignment{c}
				\topinset{\code}{\rotatebox{90}{\ban}}{14pt}{-1pt}\vfil
				\normalsize Non-disponibilité du code
		\end{minipage}}
		\uncover<4->{\begin{minipage}{0.24\linewidth}
				\centering
				\def\stackalignment{c}
				\topinset{\gear}{\rotatebox{90}{\ban}}{14pt}{-1pt}\vfil
				\normalsize Non-disponibilité des configurations
		\end{minipage}}
	\note{Non-disponibilité du jeu de données ou \textbf{version} (pas claire) du jeu de données, 
		\textbf{mauvaise spécification ou sous-spécification} du modèle et de la \textbf{procédure d'entraînement},
		manque de \textbf{disponibilité du code} nécessaire pour exécuter les expériences, ou \textbf{erreurs} dans le code,
		\textbf{configurations} du modèle déficientes
	}
	\end{frame}
	
	\begin{frame}
		\begin{figure}
			\centering
			\includegraphics[scale=0.19]{img/cycle.png}
			\caption{\href{https://eng.uber.com/scaling-michelangelo/}{From Uber Engineering}}
		\end{figure}
	\end{frame}
	
	\begin{frame}
		\begin{figure}
			\centering
			\includegraphics[scale=0.24]{img/cycle-white.png}
			\caption{\href{https://johann.schleier-smith.com/blog/2015/08/09/need-for-agile-machine-learning.html}{The need for Agile machine learning}}
		\end{figure}
	\end{frame}
	
	\section{OK, mais comment?}
	
	\begin{frame}
		\begin{figure}
			\centering
			\includegraphics[scale=0.3]{img/cycle-2.png}
			\caption{\href{https://eng.uber.com/scaling-michelangelo/}{From Uber Engineering}}
		\end{figure}
	\end{frame}

	\begin{frame}{Version des données \& Étapes de prétraitement}
		\centering
		\uncover<1->{\begin{minipage}{0.32\linewidth}
				\centering
				\fontsize{35}{35}\selectfont\textcolor[cmyk]{0.67, 0.66, 0, 0.71}\faCopy\vfil
				\normalsize Version
		\end{minipage}}
		\uncover<2->{\begin{minipage}{0.32\linewidth}
				\centering
				\fontsize{35}{35}\selectfont\textcolor[cmyk]{0.67, 0.66, 0, 0.71}\faServer\vfil
				\normalsize Gestion des versions
		\end{minipage}}
		\uncover<3->{\begin{minipage}{0.32\linewidth}
				\centering
				\fontsize{35}{35}\selectfont\textcolor[cmyk]{0.67, 0.66, 0, 0.71}\faRoad\vfil
				\normalsize Étapes prétraitement
		\end{minipage}}
		\note{
			Avec quelle version des données doit-on travailler?
			
			Comment gérer \textbf{facilement} plusieurs versions des données?
			
			Comment définir \textbf{facilement} les étapes de prétraitement des données?
			
			Il nous faut des \textit{\textbf{data pipelines}}, des tuyaux que nous pouvons \textbf{facilement} raccorder à nos modèles pour l'entrainement et la mise en production.
		
			Il faut somewhat voir nos données comme du code et qu'à chaque fois qu'on fait du traitement sur notre data c'est comme modifier notre codebase et c'est impératif d'avoir une trace, par exemple ...
		}
		
	\end{frame}

	\begin{frame}{Version des données \& Étapes de pré procession}
		\begin{itemize}
			\item Data Version Control (\href{https://dvc.org/}{DVC})
			\item Dask \cite{dask}
		\end{itemize}
	\end{frame}
	
	\begin{frame}{Code}
		\centering
		\uncover<1->{\begin{minipage}{0.32\linewidth}
				\centering
				\fontsize{35}{35}\selectfont\textcolor[cmyk]{0.67, 0.66, 0, 0.71}\faCopy\vfil
				\normalsize Version
		\end{minipage}}
		\uncover<2->{\begin{minipage}{0.32\linewidth}
				\centering
				\fontsize{35}{35}\selectfont\textcolor[cmyk]{0.67, 0.66, 0, 0.71}\faList\vfil
				\normalsize Différence
		\end{minipage}}
		\uncover<3->{\begin{minipage}{0.32\linewidth}
				\centering
				\fontsize{35}{35}\selectfont\textcolor[cmyk]{0.67, 0.66, 0, 0.71}\faCodeFork\vfil
				\normalsize Divergences
		\end{minipage}}
		\note{
			Avec quelle version du code doit-on travailler?
			
			Comment savoir \textbf{rapidement} qu'elle est la différence d'implémentation entre deux versions du modèle?
			
			Comment gérer \textbf{facilement} les divergences d'expérimentations?
			
			Il nous faut un outil nous permettant de \textbf{visualiser} la différence entre des fichiers de code et nous permettant d'avoir \textbf{plusieurs} versions du code en même temps, par exemple ...
		}
	\end{frame}

	\begin{frame}{Code}
		\begin{itemize}
			\item \href{https://git-scm.com/}{Git}
		\end{itemize}
	\end{frame}
	
	\subsection{Développement}
	\begin{frame}{Développement des modèles}
		\def\ban{\fontsize{65}{65}\selectfont\textcolor[cmyk]{0.67, 0.66, 0, 0.71}\faBan}
		\def\tool{\fontsize{35}{35}\selectfont\textcolor[cmyk]{0.67, 0.66, 0, 0.71}\faWrench}
		\centering
		\uncover<1->{\begin{minipage}{0.32\linewidth}
				\centering
				\def\stackalignment{c}
				\topinset{\tool}{\rotatebox{90}{\ban}}{14pt}{-1pt}\vfil
				\normalsize Réinventer
		\end{minipage}}
		\uncover<2->{\begin{minipage}{0.32\linewidth}
				\centering
				\fontsize{35}{35}\selectfont\textcolor[cmyk]{0.67, 0.66, 0, 0.71}\faMinusSquareO\vfil
				\normalsize Simplification
		\end{minipage}}
		\uncover<3->{\begin{minipage}{0.32\linewidth}
				\centering
				\fontsize{35}{35}\selectfont\textcolor[cmyk]{0.67, 0.66, 0, 0.71}\faFastForward\vfil
				\normalsize Facilite
		\end{minipage}}
		\note{
			Ne pas réinventer la roue.
			
			Simplifier l'écriture de code pour développer des modèles.
			
			Qui facilite l'entrainement (GPU, multi-GPU/CPU).
			
			Il nous faut des outils nous permettant de \textbf{simplifier} le \textbf{développement} de nos modèles, par exemple ...
		}
	\end{frame}

	\begin{frame}{Développement des modèles}
		\begin{itemize}
			\item Poutyne \cite{poutyne}
			\item PyTorch Lightning \cite{falcon2019pytorch}
			\item Scikit-Learn \cite{sklearn_api}
			\item Gensim \cite{rehurek_lrec}
			\item Allen NLP \cite{Gardner2017AllenNLP}
		\end{itemize}
	\end{frame}
	
	
	\begin{frame}{Entraînement, configuration et résultats}
		\centering
		\uncover<1->{\begin{minipage}{0.24\linewidth}
				\centering
				\fontsize{35}{35}\selectfont\textcolor[cmyk]{0.67, 0.66, 0, 0.71}\faCodeFork\vfil
				\normalsize Version de l'entraînement
		\end{minipage}}
		\uncover<2->{\begin{minipage}{0.24\linewidth}
				\centering
				\fontsize{35}{35}\selectfont\textcolor[cmyk]{0.67, 0.66, 0, 0.71}\faAreaChart\vfil
				\normalsize Résultats
		\end{minipage}}
		\uncover<3->{\begin{minipage}{0.24\linewidth}
				\centering
				\fontsize{35}{35}\selectfont\textcolor[cmyk]{0.67, 0.66, 0, 0.71}\faCog\vfil
				\normalsize Visualisation
		\end{minipage}}
		\uncover<4->{\begin{minipage}{0.24\linewidth}
				\centering
			\fontsize{35}{35}\selectfont\textcolor[cmyk]{0.67, 0.66, 0, 0.71}\faTimesCircleO\vfil
			\normalsize Erreurs d'entraînement
	\end{minipage}}

		\note{
			Avec quelle version du code, du modèle et des données avons-nous fait cet entrainement?
			
			Quels sont les résultats?
			
			Comment visualiser \textbf{rapidement} les résultats et les paramètres de configuration?
			
			Il nous faut des outils nous permettant de \textit{\textbf{logger}} les \textbf{paramètres} d'entrainement et les \textbf{résultats}, par exemple, ...
		
	}
	\end{frame}

	\begin{frame}{Entraînement, configuration et résultats}
		\begin{itemize}
			\item MLFlow \cite{Zaharia2018AcceleratingTM}
			\item Hydra \cite{Yadan2019Hydra}
			\item Sacred \cite{sacred}
			\item Notif \cite{notif}
		\end{itemize}
		
	\end{frame}

	\begin{frame}{Rapport et analyse des résultats}
		\centering
		\uncover<1->{\begin{minipage}{0.32\linewidth}
				\centering
				\fontsize{35}{35}\selectfont\textcolor[cmyk]{0.67, 0.66, 0, 0.71}\faTable\vfil
				\normalsize Tableau des résultats
		\end{minipage}}
		\uncover<2->{\begin{minipage}{0.32\linewidth}
				\centering
				\fontsize{35}{35}\selectfont\textcolor[cmyk]{0.67, 0.66, 0, 0.71}\faCalendar\vfil
				\normalsize Mise à jour
		\end{minipage}}
		\uncover<3->{\begin{minipage}{0.32\linewidth}
				\centering
				\fontsize{35}{35}\selectfont\textcolor[cmyk]{0.67, 0.66, 0, 0.71}\faAreaChart\vfil
				\normalsize Visualisation configuration
		\end{minipage}}
		
		\note{
			Comment créer des tableaux de résultats \textbf{facilement} (pas à la \textit{mitaine})?
			
			Comment s'assurer \textbf{facilement} que les résultats sont à jours?
			
			Comment visualiser \textbf{rapidement} les résultats et les paramètres de configuration?
			
			Il nous faut des outils nous permettant de \textbf{créer} des tableaux de résultats \textbf{à même les résultats}, soit de diminuer le plus possible le travail manuel, par exemple ...
		}
	\end{frame}


	\begin{frame}{Rapport et analyse des résultats}
		\begin{itemize}
			\item \href{https://github.com/jsleb333/python2latex}{Python2LaTeX}
			\item \href{https://www.tensorflow.org/tensorboard}{TensorBoard}
			\item \href{https://fr.wikipedia.org/wiki/Markdown}{Markdown}
		\end{itemize}
	\end{frame}
	
	\begin{frame}{Environnement}
		
		\centering
		\uncover<1->{\begin{minipage}{0.49\linewidth}
				\centering
				\fontsize{35}{35}\selectfont\textcolor[cmyk]{0.67, 0.66, 0, 0.71}\faFileArchiveO\vfil
				\normalsize Différents environnements
		\end{minipage}}
		\uncover<2->{\begin{minipage}{0.49\linewidth}
				\centering
				\fontsize{35}{35}\selectfont\textcolor[cmyk]{0.67, 0.66, 0, 0.71}\faRecycle\vfil
				\normalsize Réutilisation
		\end{minipage}}
		
		\note{
			Comment s'assurer que nos modèles fonctionnent sur d'autres environnements?
			
			Comment faciliter la réutilisation de notre code?
			}
	\end{frame}

	\begin{frame}{Environnement}
		\begin{itemize}
			\item \href{https://www.docker.com/}{Docker}
		\end{itemize}
	\end{frame}
		
	\section{La suite}
	\begin{frame}
		\centering
		\fontsize{35}{35}\selectfont\textcolor[cmyk]{0.67, 0.66, 0, 0.71}\faRefresh\vfil
		\normalsize Itérations d'expérimentations
		\note{Développer des processus rigoureux (par essais, erreurs et journaux) et ne pas prendre tout ce qui a été discuté ici comme l'unique solution.}
	\end{frame}
	
	\begin{frame}{Pour aller plus loin}
		\begin{itemize}
			\item \href{https://www.oreilly.com/library/view/clean-code-a/9780136083238/}{Clean code}
			\item \href{https://github.com/iterative/cml}{Continuous Machine Learning}
			\item Faire des tests!
			\item Writing Code for NLP Research \cite{gardner-etal-2018-writing}
			\item \href{https://www.youtube.com/watch?v=t86v3N4OshQ&list=LLFp5G_2HoipBrGaw9iAcPPw&index=693}{SOLID}
			\item Cet article \cite{pineau2020improving}
		\end{itemize}
	\end{frame}
	
	\begin{frame}
		\frametitle{Période de questions}
		
		\centering
		\fontsize{100}{100}\selectfont\textcolor[cmyk]{0.67, 0.66, 0, 0.71}
		\faQuestion
		
	\end{frame}

	%%% Copyright (C) 2020 David Beauchemin
%%%
%%% Ce fichier et tous les fichiers .tex dont la racine est
%%% mentionnée dans les commandes \include et \input ci-dessous font
%%% partie du projet «Reproductibilité en apprentissage automatique
%%% - Webinaire de l'institut intelligence et données»
%%% URL
%%%
%%% Le format et le visuel est très fortement inspiré du matériel de
%%% Vincent Goulet https://gitlab.com/vigou3/webinaire-recherche-reproductible
%%%
%%% Cette création est mise à disposition selon le contrat
%%% Attribution-Partage dans les mêmes conditions 4.0
%%% International de Creative Commons.
%%% https://creativecommons.org/licenses/by-sa/4.0/

%% Normes de présentation visuelle 2018
%%
%% - grille de 8 unités de haut
%% - 1 mesure = 1/8 d'unité
%% - bande identitaire de 1 mesure placée au bas de la 7e unité
%% - logo haut de 4 mesures avec blancs de deux mesures en haut et
%%   en bas
%% - blanc équivalent à la largeur du blason à droite du logo
%% - bande or de la largeur du logo + blanc à droite
%%
%% Dimensions du logo UL
%%
%% hauteur: 129
%% largeur totale: 312
%% largeur blason: 102
%% valeur clé: (312 + 102)/129 = 3.209302
%%
%% Dimensions de l'image
%%
%% hauteur: 55 mesures - 1pt (filet) = 54.9191919 mesures
%% largeur: 160mm
%% ratio largeur/hauteur: 160/77.23

\begingroup
\TPGrid{16}{64}
\textblockorigin{0mm}{0mm}
\setlength{\parindent}{0mm}
\setlength{\imageheight}{54.9191919\TPVertModule}
\setlength{\logoheight}{4\TPVertModule}
\setlength{\bandeorwidth}{3.209302\logoheight}
\setlength{\banderougewidth}{\paperwidth}
\addtolength{\banderougewidth}{-\bandeorwidth}
\setlength{\bandeorheight}{\TPVertModule}
\setlength{\banderougeheight}{\TPVertModule}
\setlength{\textwidth}{\paperwidth}
\addtolength{\textwidth}{-2\TPHorizModule}

\def\titlefmt{%
  \bfseries\fontsize{24}{24}\selectfont%
  MERCI DE VOTRE \\ ÉCOUTE!\par}
\def\webinaire{%
  \OverpassSemiBold\bfseries\fontsize{18}{18}\selectfont
  WEBINAIRE}
\def\fsgiid{%
	\OverpassLight\fontsize{4.5}{4.5}\selectfont%
	\bfseries\textcolor{black}{Faculté des sciences et de génie} \\
	\mdseries\textcolor[gray]{0.35}{Institut Intelligence et Données (IID)}}

\def\iidurl{%
	\OverpassLight\fontsize{4.5}{4.5}\selectfont%
	\bfseries\textcolor{black}{iid.ulaval.ca}}

%%%
%%% Page de titre
%%%
\begin{frame}[plain]
  %% bandeau identitaire
  \begin{textblock*}{\paperwidth}[0,1](0mm,56\TPVertModule)
    \textcolor{rouge}{\rule{\banderougewidth}{\banderougeheight}}% % bande rouge
    \textcolor{or}{\rule{\bandeorwidth}{\bandeorheight}}           % bande or
  \end{textblock*}

  %% logo UL
  \begin{textblock*}{\bandeorwidth}(\banderougewidth,58\TPVertModule)
    \includegraphics[height=\logoheight,keepaspectratio=true]{%
      img/ul_p}
  \end{textblock*}
  \begin{textblock*}{3\TPHorizModule}[1,0.5](14\TPHorizModule,59.7\TPVertModule)
    \raggedleft\fsgiid
  \end{textblock*}
  \begin{textblock*}{3\TPHorizModule}[0,0.5](0.7\TPHorizModule,59.7\TPVertModule)
	\iidurl
\end{textblock*}

  %% image droite
  \begin{textblock*}{0.560261\imageheight}[1,0](147mm,0mm)
    \includegraphics[height=\imageheight,keepaspectratio, angle=180,origin=c]{%
	img/degrade-coin-rouge}
  \end{textblock*}

  %% identifiant «webinaire»
  \begin{textblock*}{2\TPHorizModule}(0.7\TPHorizModule,5\TPVertModule)
    \textcolor[rgb]{0.37,0.37,0.37}{\webinaire}
  \end{textblock*}

  %% titre
  \begin{textblock*}{10\TPHorizModule}(0.7\TPHorizModule,21\TPVertModule)
    \textcolor[cmyk]{0.67, 0.66, 0, 0.71}{\titlefmt}
  \end{textblock*}\end{frame}

\endgroup

%%% Local Variables:
%%% mode: latex
%%% TeX-engine: xetex
%%% TeX-master: "webinaire-recherche-reproductible"
%%% End:

	
	\begin{frame}[t, allowframebreaks]
		\frametitle{References}
		\bibliographystyle{apalike}
		\bibliography{RAA}
	\end{frame}
	
	
	
\end{document}