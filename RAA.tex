\documentclass{beamer}
\usefonttheme[onlymath]{serif}

\usepackage[utf8]{inputenc}
\usepackage[T1]{fontenc}
\usepackage{lmodern}
\usepackage[francais]{babel}

% For table
\usepackage{booktabs} % To thicken table lines
\usepackage{multirow}
\usepackage{amsmath}

\usepackage[group-separator={,}]{siunitx}

  %for tikz figure
\usepackage{tikz}
\usetikzlibrary{decorations.pathreplacing}
\usetikzlibrary{fadings}
\usetikzlibrary{positioning} %for lstm
\usetikzlibrary{shapes, backgrounds}
\usetikzlibrary{arrows}
\usetikzlibrary{chains}
\tikzstyle{line} = [draw, -latex']
\usepackage{adjustbox}

\usepackage{pgfplots}
\usepackage{pgfplotstable}

\definecolor{color3}{rgb}{0,0.7,0.3}
\definecolor{color1}{rgb}{0,0.1,0.8}
\definecolor{color2}{rgb}{0.9,0.0,0}

\graphicspath{{img/}}


\mode<presentation> {
	\usetheme{ulaval}
	\setbeamercovered{invisible}
}

\logo{
	\includegraphics[height=0.65cm, keepaspectratio]{graal.pdf}\hspace{.2cm}\vspace{.85\paperheight}}


\title{Reproductibilité en apprentissage automatique}
%\subtitle[]{}

\author[D. Beauchemin]{David Beauchemin}
\institute[Université Laval]
{
	Département d'informatique et de génie logiciel, \\
	Université Laval\\
	\medskip
	{\emph{david.beauchemin.5@ulaval.ca}}
}
\date{30 octobre 2020}

\AtBeginSection[]
{
	\begin{frame}<beamer>
		\frametitle{Plan}
		\tableofcontents[currentsection]
	\end{frame}
}

\begin{document}
	
	
	\begin{frame}[label=titre, plain]
		\titlepage
		\begin{center}
			\includegraphics[height=1cm]{graal}
			\includegraphics[height=1cm]{UL_P}
		\end{center}
	\end{frame}

	\begin{frame}{Objectifs de la présentation}
		\begin{itemize}
			\item Sensibiliser sur les enjeux de la reproductibilité.
			\item Inciter l'intégration des solutions permettant une meilleure reproductibilité dans vos solutions d'affaires ou académiques.
		\end{itemize}
	\end{frame}

	\begin{frame}{Mes qualifications}
		\begin{itemize}
			\item Introduit (informellement) à la recherche reproductible en 2016 (RMarkdown et Git).
			\item Participation à REPROLANG, visant la reproductibilité d'articles scientifiques ayant mené à la publication d'un article en 2019.
			\item Développement en cours de solution d'intégration facilitant la reproductibilité (Poutyne, MLFlow callback).
			\item Candidat au doctorat en informatique en apprentissage automatique.
			\item Membre fondateur de Baseline.
		\end{itemize}
	\end{frame}
	
	\section{Introduction}
	\begin{frame}{C'est quoi la reproductibilité ?}
		
	\end{frame}
		

	\section{Les barrières à la reproductibilité}
	
	\section{La reproductibilité jusqu'où?}
	
	\section{La suite}
	
	
	\begin{frame}[t, allowframebreaks]
		\frametitle{References}
		\bibliographystyle{apalike}
		\bibliography{RAA}
	\end{frame}
	
\end{document}